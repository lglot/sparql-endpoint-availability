\documentclass[8pt]{beamer}
\usepackage[utf8]{inputenc}

\usepackage{amsmath}
\usepackage{setspace}


\renewcommand{\baselinestretch}{1.5}
\definecolor{darkred}{rgb}{0.6,0,0}
\definecolor{gold}{rgb}{1.0, 0.84, 0.0}

\mode<presentation>{
    \usetheme{AnnArbor}
    %\usecolortheme{beaver}
    \setbeamercolor{titlelike}{parent=structure,fg=white,bg=darkred}
    \setbeamercolor{palette primary}{fg=darkred!60!black,bg=gray!30!white}
    \setbeamercolor{palette secondary}{fg=darkred!70!black,bg=gray!15!white}
    \setbeamercolor{palette tertiary}{bg=darkred!80!black,fg=gray!10!white}
    %\setbeamercolor{palette quaternary}{fg=darkred,bg=gray!5!white}
    \setbeamercolor{frametitle}{bg=darkred}
    \setbeamercolor{section in toc}{fg=black}
    \setbeamercolor{subsection in toc}{fg=red}
    \setbeamercolor{item projected}{bg=darkred, fg=white}
    \setbeamercolor{block body}{bg=red!30,fg=black}
    \setbeamercolor{block title}{bg=purple, fg=white}
    \setbeamertemplate{caption}[numbered]
    \setbeamertemplate{itemize item}{\color{purple}$\blacksquare$}
    \setbeamertemplate{itemize subitem}{\color{purple}$\blacktriangleright$}
}


\title{Sparql Endpoint Availability}
\subtitle{Presentazione del progetto di Ingegneria del Software Avanzata}
\author{Luigi Lotito}
\date{A.A. 2021/2022}

\begin{document}



\begin{frame}
    \titlepage
\end{frame}

\begin{frame}
    \frametitle{Introduzione}
    \begin{itemize}
        \item \textbf{Sparql Endpoint Availability} è un'applicazione web
              che permette di monitorare la disponibilità
              degli SPARQL Endpoint, offrendo un'interfaccia web e un'API REST.
        \item Gli SPARQL endpoint, elemento fondamentale in ambito web semantico, rappresentano un punto d'accesso alle basi di conoscenza (knowledge base).
        \item I servizi che offrono SPARQL endpoint attualmente sono diverse centinaia, ma non sempre risultano attivi.
        \item Nasce quindi la necessità di avere un punto di riferimento dove poter visualizzare quali servizi offrono basi di conoscenza con degli SPARQL endpoint, e quali di questi sono disponibili.
    \end{itemize}
\end{frame}

\begin{frame}{Requisiti}
    Requisiti funzionali primari:
    \begin{itemize}
        \item Avere un'interfaccia web dove poter visualizzare la disponibilità attuale, giornaliera e settimanale degli endpoint.
        \item Creare un'API REST per poter ottenere le informazioni sullo stato (disponibilità) degli endpoint.
    \end{itemize}
\end{frame}

\begin{frame}{Requisiti}
    Requisiti funzioniali derivati:
    \begin{itemize}
        \item L'applicazione deve storicizzare gli endpoint, di cui si vuole monitorare la disponibilità.
        \item Ogni ora, l'applicazione deve inviare una richiesta HTTP, contente una semplice query SPARQL, ad ogni SPARQL endpoint memorizzato nel sistema. Se l'endpoint risponde, allora si considera che sia disponibile.
        \item Gli esiti delle richieste devono essere storicizzati, in modo da poter visualizzare la disponibilità degli endpoint attuale e nel tempo.
    \end{itemize}
    Security:
    \begin{itemize}
        \item I dati degli vanno protetti da accessi non autorizzati.
        \item L'accesso all'interfaccia web deve essere protetto da autenticazione.
        \item Per l'API REST, l'accesso deve essere protetto da autenticazione JWT.
    \end{itemize}
\end{frame}

\begin{frame}{Utenti e Amministratori}
    Per poter utilizzare l'apllicazione, è necessario registrarsi come utente, o disporre
    di un account di amministratore.
    \begin{itemize}
        \item Gli utenti possono visualizzare lo stato degli endpoint tramite interfaccia web ed API REST, ma non hanno i permessi di scrittura.
        \item Gli utenti amministratori (oltre ad avere gli stessi permessi degli utenti) devono poter aggiungere, aggiungere o rimuovere degli endpoint dal sistema, tramite interfaccia web, o tramite API REST.
        \item Gli utenti amministratori devono poter disabilitare o eliminare gli utenti tramite interfaccia web.
    \end{itemize}
\end{frame}

\begin{frame}{Tecnologie utilizzate}
    \begin{itemize}
        \item \textbf{Framework}: Spring
        \item \textbf{Frontend}: Thymeleaf, Bootstrap
        \item \textbf{Backend}: Java, Spring Boot, Spring Security, Spring Data JPA, Apache Jena
        \item \textbf{Develop Tools}: \underline{Maven}, \underline{Git}
        \item \textbf{Database}: PostgreSQL, MySQL, H2
        \item \textbf{Testing}: \underline{JUnit}, \underline{Mockito}, \underline{Selenium}
        \item \textbf{CI/CD}: \underline{GitHub Actions}
        \item \textbf{Hosting}: \underline{Docker}, Docker Compose, \underline{Heroku}
        \item \textbf{API DOC}: Swagger
    \end{itemize}
\end{frame}

\begin{frame}[allowframebreaks]
    \frametitle{Testing}
    Sono state testate le seguenti funzionalità:
    \begin{itemize}
        \item Il corretto funzionamento delle API REST, in particolare è stato testato
              che che ogni metodo REST ritorni il risultato atteso anche in funzione del ruolo dell'utente che effettua la richiesta.
              Sono stati fatti sia test di unità del metodo Java che si occupa di gestire la richiesta, sia test di sistema.\\
              Si possono visualizzare le API su \url{https://sparql-endpoint-availability.herokuapp.com/api-doc}.
        \item Il corretto funzionamento dell'interfaccia web, tramite \textbf{Selenium}.\\
              I browser supportati sono Chrome , Firefox ed Edge.
              Le funzionalità testate sono:
              \begin{itemize}
                  \item Registrazione ,login e logout di un utente.
                  \item Visualizzazione corretta delle informazioni del profilo utente.
                  \item Visulizzazione pagina con la disponibilità degli endpoint.
                  \item Visualizzazione degli utenti registrati da parte di un utente amministratore, e possibilità di disabilitare o eliminare un utente.
              \end{itemize}
        \item Auntenticazione JWT
        \item La formattazione dei dati riguardanti la disponibilità degli endpoint
        \item Alcuni metodi di interfacciamento con il database, in particolare quelli che si occupano di salvare e recuperare le informazioni riguardanti gli endpoint.
    \end{itemize}
\end{frame}




\end{document}